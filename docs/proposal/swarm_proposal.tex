\documentclass[]{article}

\usepackage{xargs} 
\usepackage[colorinlistoftodos,prependcaption,textsize=tiny]{todonotes}
\newcommandx{\unsure}[2][1=]{\todo[linecolor=red,backgroundcolor=red!25,bordercolor=red,#1]{#2}}
\newcommandx{\change}[2][1=]{\todo[linecolor=blue,backgroundcolor=blue!25,bordercolor=blue,#1]{#2}}
\newcommandx{\info}[2][1=]{\todo[linecolor=OliveGreen,backgroundcolor=OliveGreen!25,bordercolor=OliveGreen,#1]{#2}}
\newcommandx{\improvement}[2][1=]{\todo[linecolor=Plum,backgroundcolor=Plum!25,bordercolor=Plum,#1]{#2}}
\newcommandx{\thiswillnotshow}[2][1=]{\todo[disable,#1]{#2}}

%opening
\title{Command Language for Single-User, Multi-Robot Search}
\author{Abraham Shultz}

\begin{document}

\maketitle

\begin{abstract}

\end{abstract}

Methods for command and control that are based on issuing individual orders to individual actors, whether they are robotic or human, do not scale to large numbers of actors \unsure{Cite this}. 
By defining a mapping from user interface gestures to individual programs loaded on each robot, we can allow an individual to control arbitrarily large, heterogeneous swarms.
In order to remain robust in the face of failure, the overall action of the swarm should be decentralized emergent behavior, rather than a centralized orchestration \unsure{Cite this}. 
Each robot receives its own program, and the sum of the execution of the programs on each robot results in task completion as emergent property.

\section{Research Statement}

\subsection{Problem Statement}

One potential method to control a swarm is having a central computer dictate to individual robots how the robots should move.
However, centralized control is only as robust as the central controller. 
Distributed control systems do not have the single point of failure that centralized models have. 
In order to create reliable and useful swarm robotic systems, users must be able to specify emergent behaviors that the swarm can converge to without reliable orchestration from a central controller. 

The current state of development of emergent control of swarms is guided by ad-hoc, iterative development models that are somewhat suited to software developers, but not suited to use by non-programming end users \cite{palmer2005behavioral}.
The motivating examples of uses for swarms are task oriented, such as sending swarm robots into disaster zones to search for survivors. 
While first responders are frequently trained in many specialized skills to perform their jobs, adding programming swarm robots to their training would be impractical. 
Even if first responders could be expected to develop software in a disaster zone, the situation frequently develops faster than the build/test cycles of software development can react. 
It is desirable to automate the construction of control software for a swarm so that it can adapt to a situation, without requiring significant development time. 
In order to support interactive control during a developing situation, the construction of the software should occur over a similar time scale to the user interactions. 

In order to remain useful in real-world applications, this work makes certain assumptions about the robots and the swarm. 
Networking between the robots is expected to be unreliable. 
Individual robots have limited transmission power over radio or infrared light links. 
Network links are also frequently attenuated by distance or intervening objects. 
As a result, when robots spread out into an area, some robots may become unreachable from others. 
Any software that purports to control a swarm under these conditions cannot rely on perfect connectivity for its operation. 
Robots perception of their environment is frequently limited. 
Objects can block the sensing field of sensors.
Even without obstructions, most sensors have a limited effective range. 
Because the real world is dynamic, robots can make only limited assumptions about what they cannot sense, and can only sense a limited area. 
Algorithms to control robots must primarily make use of locally sensed information, and secondary make use of information received in communications from other robots. 
Because of the limited range of their sensing and networking, the localization of the robots may also be unreliable. 
GPS provides a global coordinate system that robots can easily localize in, but GPS signal is absent indoors, under dense tree cover, and in many other areas where it would be desirable to use swarm robots. 
The GPS signal is also weak, and so can easily be jammed by ambient RF interference or by malicious actors. 
Robots can also fail. 
The harsh conditions of the WTC rubble pile destroyed many of the robots sent in \cite{Micire02analysisof}.
In order to remain robust in the face of failures, algorithms should not be developed to depend on the perfect functioning of any individual robot. 
Rather, the behavior should emerge from the interaction of multiple, interchangeable robots, and tolerate the loss of individual robots up to some limit. 
In a situation where many or all of the robots attempting to complete a task are destroyed or disabled, it would be unreasonable to expect the task to be completed, but ideally, the bounds on how many robots are required to complete the task should be knowable. 

\subsection{Hypothesis}

There exists a maximum swarm size beyond which users will transition from treating robots as individuals to treating the swarm as a whole. 
This transition point will be apparent because of a change in the gesture set that the user uses to interact with the swarm. 
It is hypothesized that above the transition point, users will be more likely to neglect some subset of the available robots. 
The user will instead use commands that control the bulk of the robots the robots as a cloud or flock, but may leave some robots unused. 
For example, the user may switch from selecting robots as individuals to shaping and pushing the swarm the way a child might play with a bug, putting their hand down so the bug goes around or avoids it, touching the back of the bug gently to make it scurry forwards, and so forth, or by shaping the group as if sculpting, with pushing and pinching to ``carry'' groups around. 
The user may also change how they indicate which robots are to be interacted with. 
Rather than selecting each robot by clicking on it, the may ``paint'' over the area containing the robots they want to use. 
The size of the swarm where changes in the user gestures occur will indicate the transition point between the user intent to interact with individual robots as opposed to interacting with the swarm as a whole. 

Once the ratio of the size of individual swarm members to the size of the area the swarm is in becomes sufficiently large, displaying the swarm members at the same scale as the map will result in the representation of the swarm members being too small to interact with. 
This problem will arise at smaller scales if the swarm robots are themselves quite small, and since making swarm robot smaller is an active area of research \unsure{Find an overview paper, cite that}, this problem will only get worse. 
Scaling the representation of the robots up, relative to the map, will make the robot representations overlap unrealistically and obscure the map. 
Instead, we propose that for certain scales of swarms, it makes sense to represent the swarm as the area covered, rather than the locations of the individual robots.
This approach has been used successfully for navigation in three dimensions, by developing a controller that causes the individual UAVs to remain within a bounding prism, and allowing the user to control the shape and location of that prism \cite{ayanian2014controlling}.
Altering how the user interface displays the location of the robots in the swarm will affect the transition point. 
More specifically, a display which obscures individual robots and displays a cloud or swarm boundary will cause the user to treat the swarm as a whole rather than individuals, which will be apparent because the user will use different gestures. 

It is possible to convert user commands to programs which can be transmitted to the robots and will converge to the desired behavior using only local sensing and local communications, and without resorting to global, absolute localization. %This is stupidly broad. Possible may not be feasible, or may have weird pathologies.


\subsection{Approach}

For the purposes of this work, the tasks that users will be asked to complete consist largely of directing the motion of the swarm. 
While a task is generally defined as a unit of work to be accomplished, motion within space is not usually considered a task. 
Instead, motion within a space is usually performed as part of a task.
For example, searching an area requires moving over the area to be searched.
However, specialized tasks will likely require specialized interfaces to provide the information that is specific to that task, while this work intends to discover a more general set of commands that can be used across many tasks. 

In order to create a mapping from commands, as issued by the users, to a set of individual robot programs for multi-robot command and control, the gesture set for the commands must first be specified. 
Multitouch gesture sets as a command language for controlling robots have been developed by an empirical process with naive users \cite{Micire:2009:ANG:1731903.1731912}. 
These gestures frequently consisted of sequences of gestures that roughly fit the linguistic structure of a sentence, with the first gesture indicating the subject of the sentence, and the next gesture indicating a verb and possibly an object. 
As a consequence, the gestures form a sort of language, and commands are sentences in the gesture language.
A user might select a group of robots by circling them. 
These robots would be the subject of the ``sentence''.
Then the user might draw a path or tap a spot on a map, indicating that the robots should ``go here''. 
Taken all together, the sentence could be read as ``these robots, follow this line to this location'', with the pronouns disambiguated by the locations of the command gesture on the screen.

One way to define such an interface would be to select a fixed set of gestures, and then train the users to use those gestures when they interact with the system. 
However, if the system is not one that the users use frequently, the will forget the training. 
Since the advent of multitouch interfaces for smartphones and the trackpads of some laptops, many users already have some prior experience with multitouch gesture controls in everyday life. 
Multitouch gestures can also be imitations of the way the users would expect to interact with material objects. 
For example, zooming out of a map view by pinching two fingers together imitates the distant points of the map becoming closer together, indicating outward zoom, and spreading the fingers imitates stretching a smaller region to cover the screen for an inward zoom. 
From these ``naturalistic'' expectations and daily use, users already have some idea of how a multitouch user interface can work. 
If the interface fulfills these intuitions, the users will find it easier to learn to use. 

For the purposes of this research, an intuitive gesture language is one that is freely chosen by the majority of users. 
In \cite{Micire:2009:ANG:1731903.1731912}, for each available command, one or two gestures were used by 60\% or more of the users. 
These gestures are the intuitive gestures for issuing the command. 
It is possible that there is no intuitive gesture for a given task.
If no two users use the same gesture for the same commands, or, more generally, there is very poor inter-user similarity for the gestures chosen to issue a command, then there is not an intuitive gesture for that command.

In order to determine if the intuitive gestures change with the size of the swarm, tests will be conducted with varying swarm sizes performing the same tasks. 
The swarm used for these tests will be large, with the lower bound on its size being significantly larger than the number of fingers a user could potentially gesture with. 
In order to have a less tongue-in-cheek definition of ``large'', the scale required for a swarm to be considered ``large'' will be determined empirically.
It is expected that there exists a transition point for the number of members in a swarm where users will stop interacting with the UI representation of the members of the swarm as individuals, and attempt to interact with the representations as groups or collections. 
A large swarm is, then, a swarm with a number of members above the point at which such a transition occurs. 
 
All of the valid expressions possible in the command language should be converted into programs for the robots, or the user must be usefully informed as to why it was not possible. 
The synthesized program should result in convergence of the swarm's overall behavior to the desired result. 
Clearly, in a developing situation in the real world, success may be impossible, and so there is not a practical way to guarantee that a particular valid command sequence will result in a particular desired state of the world. 
However, certain minimum bounds on the problem may be able to be used to determine if a desired task is certain to fail. %TODO STAY OUT OF GRIDWORLD FORMAL GROTESQUERY

\subsection{Swarm Robot Support Software}

The individual robots being developed for this research do not have very powerful processors. 
The majority of the processing will be performed on a host computer running the ROS software framework. 
Each robot's processor will mostly be concerned with interfacing with the robot's sensors, if any, and controlling the motors of the robot. 

In order to locate the robots within the experiment area, an overhead camera will be used to detect machine-readable symbols on each robot. 
The symbols provide location, heading, and a unique identifier for each robot. 
Because the system has an omniscient-view camera, other objects in the robot arena can also be tracked. 
For example, obstacles can be created by drawing lines on the floor of the robot enclosure. 
Different colors could represent different types of obstacles, or qualities of the obstacles that are relevant to the software under test.

Since the robots will all be reporting to a central server, and the central server will also be receiving the video from the overhead camera, it may appear that this is a highly centralized system. 
However, the central computer will provide a framework for implementing a decentralized control scheme on the individual robots. 
Rather than controlling each robot, the central computer will maintain a separate process for each robot in the swarm. 
Each of these robot processes will only have access to the information that would be available to that robot, and so will act as a local control program for the robot, but will have the full processing resources of the host computer. 
As a result, the individual robots can be small, lightweight, and consume relatively little electrical power, but the system as a whole will endow them with significant computing power. 

Similarly, it should be stressed that while the central computer can localize the robots, both relative to each other and by absolute position within the arena, this information may be withheld from the individual robots. 
The code virtually operating on the robot may be neither aware of its own position in the world, or the location of other robots, if the experiment calls for such a lack of information. 
However, the central computer can use the location information to create ``virtual sensors'' for each robot. 
The sensor precepts from virtual sensors would be simulated, but their magnitude or direction may be based on the location of a real robot relative to a real object in the experiment area. 
For example, collision avoidance between robots A and B could be implemented by a virtual sensor on each robot that indicates the direction and heading of the nearest robot. 
Since the central computer knows the location of each robot, the relevant information can be sent to each robot's control process as if it were coming from a sensor on the robot. 
The virtual sensors can also be configured to emulate error conditions such as noisy sensors, failed sensors, degraded localization, and so forth.
Virtual parameter tweaking will allow fine-grained testing of the behavior of algorithms under imperfect conditions, and the response of human users to unreliability in the swarm. 

If the robots are required to communicate directly with each other, the communication will pass through a virtual network. 
From the point of view of the robots, messages sent into the virtual network will be delivered to other robots as if the messages were sent directly from one robot to another. 
However, all the communication will be taking place between processes running on the central computer.
By changing how the messages are delivered by the central system, we can implement full connectivity, range-limited mesh networking, directional beacons, or other forms of networking. 
We can also vary the reliability of the network, by dropping some messages or reducing parameters based on elements of the virtual environments. 
For example, signals that pass through a virtual wall may have a reduced virtual signal strength and range, or may not arrive. 
Having communication mediated by software on the host will allow for simple experimentation with variable network bandwidth and reliability. 

Because children's toys are manufactured to fairly low tolerances, the correlation between motion commands issued by the control software and resulting motions of the robot are prone to error. 
The robots are also intended to be heterogeneous, partly because of the advantages of heterogeneity in a swarm, and partly because toy supplies are unreliable.
While toys in the general case are expected to remain available, a particular line of toys might be discontinued or a modified version released. 
The system as a whole should be robust against the addition of new types of mobility hardware. 
Because of the selection of hardware with a WiFi module, every hardware module is programmed with a unique MAC address. 
In order to learn the control rules, the control software will send motion commands to each robot, and observe the displacement of that robot. 
By varying motion commands and observing the resulting displacement, the system will learn the effect of motion commands on each robot, and so be able to control the robots. 

\subsection{Expected Contributions}
\subsubsection{Primary Contributions}

This work will attempt to determine if there is an intuitive multitouch gesture set for a single user to command a large swarm of robots.
The influence of the presentation of the interface on the gesture set will be examined. 
The gesture set will also be analyzed to determine if the size of the swarm has any effect on the gestures used, or on neglect of individual robots by the user. 

The automatic conversion from a user-specified task into a set of command programs to be distributed to the swarm robots is still an open question.
One recent approach uses a human-in-the-loop multitouch interface to allow a human to guide a swarm by drawing a bounding prism that the swarm attempts to remain within \cite{ayanian2014controlling}. 
As the bounding prism moves, the swarm moves with it, with the individual robots performing obstacle avoidance. 
However, this work assumes that the individual swarm units can localize themselves, and that there is constant availability of communications between all swarm members and the central controller. 
For a number of reasons, these assumptions frequently fail to hold, and so a more robust system can be designed by assuming that localization and communication are difficult. 
This work will attempt to create an automated process by which user-specified behaviors of the swarm as a whole can be converted into programs that run on individual robots. 
The behavior of the individual robots under this control should converge to the user-specified behavior without further communication from the central server.

\subsubsection{\emph{En Passant} Contributions}

In chess, \emph{en passant} is a special rule for capture that can be used to capture a pawn that advances two ranks from its start position, passing an opponent pawn. 
The literal meaning is ``in passing'', and so it is used in this context to refer to potentially useful outcomes from this thesis that are not intended as primary goals, but are developed in passing, on the way to the primary goals. 

The robot swarm developed for this work consists of a hardware module for controlling two motors connected to a toy, such as a small RC car, for mobility. 
The reasons for choosing this hardware design are explained in more detail below, but the overall intent is to have an inexpensive platform available for swarm research,
without having to rely on any particular group of swarm robotics researchers starting and maintaining a side business supporting and selling robots.
Duplication of software and other digital artifacts is trivial, so constructing a duplicate of the hardware becomes the primary difficulty. 
The use of toys for the mechanical components of the robots is intended to reduce the difficulty of constructing the hardware. 
If researchers are not to be expected to become entrepreneurs, they should also not be expected to become expert machine tool operators.
The hardware resulting from this work will be designed so that it can be duplicated by a researcher using common tools, and possessed of no more than hobby-level familiarity with electronic hardware.

\section{Related Work}

Multitouch interfaces have been determined to improve on WIMP or voice interfaces for multi-robot control in a sequence of command and control tasks, including commanding the swarm to a location, performing reconnaissance, and having the swarm cross a dangerous area \cite{hayes2010multi}.
The interface displayed the locations of the robots on a directly manipulatable map, and used movable or semi-transparent user interface widgets, in order to minimize occlusion of the map. 
Areas were selected with with drawing gestures, and paths with fluid strokes, rather than e.g. selection of vertices bounding an area.
The use of multi-touch interaction is desirable because one-at-a-time selection doesn't scale beyond a very limited number of robots.
In order to interact with large groups of robots, the user must be able to perform operations on areas and groupings, rather than on the single point available with a traditional pointer-based interface. 

\subsection{Overview of Previous Swarm Hardware}

Many impressive designs for swarm robot platforms have been proposed, and constructed as part of research in swarm robotics. 
However, most of these platforms are no longer easily commercially available, or never were. 

At the low end, in terms of scale, the I-SWARM Project intended to create a 2x2x1mm robot that moved by stick-slip locomotion actuated by piezo levers\cite{seyfried2005swarm}. 
Over the course of the project from 2004-2008, the hardware was developed and used in research, but was not converted to a commercial product. 

Alice, by Caprari et al. packed a PIC16F84 processor, motors, RF and IR networking, and enough battery power for 10 hours of autonomy into a robot measuring under one cubic inch \cite{caprari1998autonomous}. 
Unfortunately, the processor is anemic by modern standards, and the platform as a whole is not commercially available anymore. 
``Development of a Miniature Robot for Swarm Robotic Application" describes the construction of a similar robot, with an updated processor, called AmIR\cite{arvin2009development}. 
There is no evidence that AmIR was ever widely available.
Similarly, the robot described in ``A Miniature Mobile Robot With a Color Stereo Camera System for Swarm Robotics Research'' combines a relatively modern microprocessor with a DSP for on-board vision processing \cite{haverinen2005miniature}. 
Again, it is not a platform that other researchers could buy.

The Jasmine swarm robots were possibly the closest thing to a successor to Alice \cite{kernbach2011swarmrobot}.
Jasmine measured 26x26x20mm, and included an ATMega processor, IR close range communication and obstacle detection, two motor skid steering, and li-po batteries.
Unfortunately, Jasmine units cost about 100 Euro each when they were available, and they are no longer available for purchase. 
The plans and information required to reproduce Jasmine units are available for free at Swarmrobot.org.
Assembling a Jasmine robot is not beyond the reach of competent electronics hobbyists, but it does require some unusual build processes, such as grinding down the cases of certain electronic parts and filling holes in the PCB with solder to prevent light leaks. 
The chassis of Jasmine is also a custom mechanical assembly, rather than a commercially available product. 

The Epuck from EFPL is approximately ~800 swiss francs per unit, so the cost of maintaining a large swarm can become daunting quickly. 
Michael Bonani's MarXbot runs into a similar problem, in that it has a strong computer and a rich set of sensors and effectors, but as a result it is quite expensive \cite{bonani2010marxbot}. 

One way to reduce the cost of swarm robots is to use commercial, off-the-shelf (COTS) hardware in the construction of the robot. 
Reusing existing hardware leverages the economies of scale that reduce the price of commercial hardware, as well as eliminating the need to design or build the COTS parts. 
Use of COTS part in research robotics has lead to at least two platforms refered to as COTSBots.
Bergbreiter's COTSBots used mote hardware for the communications link and sensing, plus a motor control add-on board  \cite{bergbreiter2003cotsbots}. 
The mobility platform is a hacked toy, in particular, a specific brand of high-quality micro RC car.
At the time of this writing, the car used in COTSBots is moderately expensive for a toy car, although quite cheap for a research robot, costing a little over \$100USD per unit. 
Bergbreiter's COTSBots use TinyOS, a modular and event-driven framework for developing node software. 
ROS also provides a modular, event-driven framework, so many of the design ideas, if not the code, are equally applicable to the system described in this paper. 
TinyOS is written in a dialect of C called nesC rather than ROS's polyglot approach. The motor and mote boards communicate using a messaging layer, again like ROS. 
The motor driver board is not commercially available, but can be custom-built by board fabrication companies, without the researcher having to assemble it by hand. 
Ohio State also developed a very small microwave RADAR that can go on the boards.

The second version of COTSBots arrived 8 years later, in ``COTSBots: Computationally Powerful, Low-Cost Robots for Computer Science Curriculums" \cite{soule2011cotsbots}. Soule and Heckendorne describe a platform composed of a laptop, which controls a modified RC car, tank, or similar toy through some combination of motor drivers. 
Due to the diversity of possible combinations of hardware that can be assembled into this configuration, it is still a very viable platform. 
However, the minimum size of this style of COTSBot is the size of a laptop, which is in turn dictated by the minimum size of a useful keyboard. 
The large size of these COTSBots demands a very large space if the density of robots in a large swarm is to be kept low. 
Additionally, each laptop has a screen, keyboard, and so forth that are not useful while the robot is operating. 
All of these parts add to the overall cost of the swarm. 

Currently existing swarm robots are too expensive to build a large swarm, with the exception of the Harvard Kilobots \cite{rubenstein2014kilobot}. 
Kilobots contain about \$15 worth of parts, but a 10-pack sells for 1100 Swiss francs, or about \$112 (US) per robot. 
However, the Kilobots do not have hardware heterogeneity beyond that produced by wear on the robots over time. 
Wear and tear on the robots will result in differences between the robots, even if they had started from identical conditions. 
Even sending part of a swarm up a hill while the rest remain at the bottom will result in the higher members of the swarm having less battery life than those that didn't move. 
Battery life may be relevant to allocation of tasks, meaning that the physical elevation gradient of the robots becomes a gradient of fitness for a task as well. 
For very small groups, individual kilobots can be programmed differently, but any attempt to give each of a very large collection of robots an unique program will take a long time. 
The kilobots also move by stick-slip motion, and so must operate on a smooth surface, such as a whiteboard. 

The previously described projects tend to fall into one of two groups, from a hardware perspective. 
The first group uses microcontrollers and very limited onboard computation, but is small and relatively cheap.
This group includes Alice, Jasmine, AmIR, and so forth.
Due to their limited computation, these systems do not support complex algorithms such as vision processing. 
The second group use more powerful computers, but at a significant cost in weight, power consumption, and financial outlay.



\section{Work Plan}

\subsection{Build a swarm}

In order to experiment with swarm robotics, a swarm must be made.
Much can be done in simulation, but it is a daunting prospect to simulate the dynamics of real robot motion and the changes in the performance of each swarm member as mechanical wear and other forces influence them. 
Further, swarm simulations must not have bugs in the software which results in incorrect behavior. 
Genetic algorithms are infamous for exploiting quirks of their simulation environment which are not present in the real world, but still allow members of the population to inflate their fitness. 
Any behavior of a physical, unsimulated swarm is the real behavior of the swarm, and cannot be blamed on a defect in the simulation of the swarm or its environment. 

\subsection{Swarm Robot Hardware}

Swarm robots are generally small. 
The reason to keep swarm robots small is two-fold. 
First, larger robots consume more materials per unit, and so costs more money.
As a result, for a given number of swarm units, larger robots will result in a higher cost swarm. 
Second, each robot requires some amount of space to move around in. 
To keep the ratio of free space to robots constant, the area of space used by the robots grows as the robots do. 
If the ratio isn't kept constant, the robots will crowd each other, and so large robots will require either a very large space, or become overly crowded.


\subsubsection{Proposed Swarm Hardware Design}

The challenge of construction of swarm robots hardware, then, is to put all of the same parts as non-swarm mobile robots: a mobility platform, a processor, some sensors, and a communication system, into a small package.
In order to be both heterogeneous and inexpensive, the robots used for this work will be constructed by developing a Commercial Off-The-Shelf (COTS) modular control hardware platform that can be attached to children's toys. 
Each controller will use a ESP-8266 wifi module for wireless communication and as a micro controller. 
The ESP-8266 costs approximately \$3-5, and contains both a wireless interface and a micro controller that can be programmed from a variety of programming environments and languages, including Lua and the Arduino variant of C/C++.   
The robots described in this work accomplish this task by using a Commercial Off-The-Shelf (COTS) module to provide Wi-Fi connectivity and a microcontroller for processing. 
The ESP8266 is avialable in several form factors, each designated by a different suffix. 
The 03 version was chosen because it offers more GPIO pins than most other versions, and includes an internal antenna, which some versions lack. 

The robots themselves are effectively Internet of Things (IoT) nodes with the ability to move around. 
This is, then, an extension of the work done with mobile sensor motes called COTSBots \cite{bergbreiter2003cotsbots}.
Modified toys are an adequate substitute for custom mechanical assemblies, and permit easy experimentation with heterogeneous swarms. 
This swarm controller module was designed to be used as a replacement for the control electronics of children's toys, similar to the Spider-Bots developed by Laird, Price, and Raptis \cite{lairdspider}. 
Most children's toys use either one motor with a mechanical linkage to cause the toy to turn when the motor is reversed, or two motors.
Two-motor toys frequently use either differential steering or have one motor provide drive power and the other provide steering. 

\subsubsection{UI Designs}

One approach to getting feedback from a swarm was the development of the Swarmish sound and light system\cite{mclurkin2006speaking}. 
Swarmish provides an ambient means of determining the overall state of the swarm, as well as some information about individual robots. 
The swarm that used Swarmish had autonomous charging, and so the individual robots had long runtimes, and minimal one-on-one interaction with humans. 
As a result, most of the interactions were remote.
The ``ambient'' aspect of the interaction is that the information is continuously available, and the human user ``tunes in'' to it when needed. 
Swarmish uses a set of colored lights and sounds, produced by each robot, to provide feedback. 
The lights were in three colors, and had a total of 108 different combinations of colors and blink sequences, as a visual indicator of the state of each robot. 
In addition to the lights, each robot could produce MIDI notes over its audio system. 
Each note can vary in instrument, pitch, duration, and volume, in addition to having tempos and rhythms as the code executes. 
The designers of Swarmish indicate that the sum of the audio output of the swarm could provide a overall idea of the status of the swarm, but that as a musical instrument, it is difficult to play well. 
Further, the use of lights as signaling mechanisms assumes that you can look at the robots. 

If we accept the assumption that the robots are visible to the user, the robots can carry some form of display that provides local information to the user. 
This information can then be displayed as an overlay in the real world, with the display of the information conterminous with its presence\cite{Daily:2003:WEI:820752.821587}. 
Local display of local information works if the user is part of a hybrid human/robot team, and so is in the same location as the users. 
However, there are many situations where the robot is not in the same location as the user. 
A common example is urban search and rescue, where buildings may be known to be unsafe, or of unknown stability, but it is desirable to search them for trapped people. 
In such a situation, the human user would rather be located elsewhere, and receive information from the robots. 

For situations where the user is not located in the same area as the robots, one possible approach is a ``call center'', where robots can request human attention when required \cite{chen2011supervisory}. The human in the call center, however, is faced with having to answer potentially multiple calls with no awareness of the robot's situation. 

One method that is frequently cited as a basis for call center UI is Supervisory Control. 
Supervisory control has the human act as the planner and monitor of the systems being supervised, but allowing the systems to operate on their own.
Automation is frequently broken down into ten levels of automation, with 10 being a fully automatic system with no humans involved, and level 1 having no automation, such as a bicycle \cite{parasuraman2000model}. 
It would be expected that reducing the number of times the human is required to interact with the robot will permit the user to operate more robots.
At level 1, the user has to interact constantly, and so could not be expected to operate more than one robot. 
By increasing the level of autonomy of the robot, the time required for the user to operate the robot decreases.  
Level 5 is a sort of operation by consent model, where the computer chooses a route and executes it if the human permits it. 
At level 5, the robot only contacts the user to confirm plans, and does not occupy the human's time while the plans are in progress. 
Level 9 is the inverse of level 5, where the computer informs the human only in exceptional cases. 
One possible approach to maintain a constant and manageable workload on the user is adapting the level of automation to the workload. 
When the load is low, the user is more directly engaged, but when load is high, there is more automated assistance. 
Adaptation does not have to be based on measured load, but could instead be based on perceived load or physiological markers in the user. 

However, in situations with even moderate numbers of robots, it may be that even relatively high levels of automation may overwhelm the user. 
Level 5 is a fairly high level, but with a large number of robots checking in, even this level may generate too many events for the human to deal with. 
As the number of robots increases further, even level nine may overwhelm the operator, despite the fact that the robots are only checking in with the operator when an exceptional situation occurs. 
Increasing the use of automation may also create new difficulties by leaving operator out of practice, or encouraging mis-placed trust in the automation's ability. 

In fact, the any kind of multitasking may prove insufficient for large swarms. 
Operator multitasking starts to fail once there are more than about 12 robots being operated at once, in the best cases \change{cite this}.
Failure generally takes the form of task effectiveness no longer increasing as more robots are added.
Instead, the amount of time the user spends interacting with the robots begins to outweigh the time that the robots are usefully active between interactions, and so the time the robots spend uselessly waiting for interactions rises \change{cite whatever defines neglect time}. 
Most of the best cases are uncrewed aerial vehicles (UAVs), which require relatively little oversight. 
Uncrewed ground vehicles (UGVs) require more oversight than UAVs, due to the higher complexity of the ground environment. 
Estimates place the limits on the number of robots under control at 12 or 13 for UAVs and 3-9 for UGVs \cite{WangSearchScale}.  
There is some latitude, at least in UGVs, to increase multitasking by increasing automation, as shown by the relatively wide range in the interaction limits, but even 9 is nowhere near the scale of kilo-robot swarms \cite{Olsen:2004:FMH:985692.985722}.
%The use of models such as developing a grounding using a proxy for the robot may assist the user in redeveloping SA \cite{stubbs2008using}. 

Ecological interface design (EID) presents a possible guide to the architecture of user interfaces for swarm robotics, and has been used in interfaces with mixed human-robot teams \cite{vicente1992ecological, gancet2010user}. 
In EID, a user's abilities that enable them to interact with a system is separated into a taxonomy of skills, rules, and knowledge. 
The user has skills, which are rote, simple activities that form the basis of the normal operation of the system. 
The user also has rules about the system. 
Rules allow the user to handle exceptions or unusual cases that have come up before. 
Rules do not require the user to understand the system, just to know that when certain situations are recognized, certain other actions must be performed in response. 
Beyond rules and skills, the user also has knowledge of the system. 
Knowledge allows the user to handle novel exceptions. 
Knowledge gives the user an understanding of how the system works, and can apply that understanding to react to situations that the user has not experienced or been told about before. 
Events are also broken into three levels: routine, which uses skills; foreseen exceptions, which use rules; and unforeseen exceptions, which use knowledge. 
All levels should be supported by the interface, but the user should not be forced to operate at a higher level than is required. 
The abstraction of the process maps onto the hierarchy of ecological design, with the highest level being the function of the process and the lowest level being how the function is accomplished. 
At each level, there are constraints on the process that are used to define the normal operation of the process.
Detection of exceptions requires the display of all constraints, because exception is the breaking of constraints, and undisplayed constraints cannot be assessed to determine if they hold.
User should be able to extract meaning from the information display quickly, as in the case of Swarmish and the robot-as-pixel UI designs.
By using the lights in Swarmish, the user can assess the state of individual robots, but by listening to the overall sound of the swarm, the user can also assess the behavior of the system as a whole. 
As the system changes, the changes and predictions should be highlighted so that the user understands consequences of their actions. 

EID is well-positioned to deal with emergent behavior, because the emergent behavior of the entire system is present at the functional level, but is composed of actions at the physical level.  
Previous work in multi-touch interfaces directly relates to EID by providing both an omniscient camera view for direct manipulation of the high-level, functional actions of the entire swarm, and the ability to move down the hierarchy to control individual swarm members \cite{Micire:2009:ANG:1731903.1731912}.
In addition, the control of swarm robots can be viewed as a hierarchy of increasing abstraction. 
At the least abstract, base level are the individual interactions of the swarm robots with each other and their environment, as dictated by their explicit programs. 
Above that level is the implicit, emergent behavior of the swarm as a whole. 
Finally, the most abstract level is the user intent, as expressed in the interface through their gestures. 
This heirarchy corresponds well to the abstraction of process in EID, with discrete physical actions at the lowest level and the overall results of the process at the highest level. 

Once interaction some section of the process has been taken over by automation, the user operates primarily in the rules and knowledge areas, dealing with exceptions \cite{vicente2002ecological}.
The interface should allow direct manipulation of perceptual forms that map directly onto work-domain constraints and represent all of the information identified by the abstraction hierarchy. 
In a swarm context, this means displaying functional information in such a way that the user can move across the hierarchy from individual swarm bots to high-level swarm-wide tasks, and interact at all levels to control the swarm. 
More practically, this means that the information displayed must be integrated in such a way that the mapping from one unit of information to another is made apparent in the interface, rather than offloaded to the user to compute in their head \cite{yanco2004beyond}. 
For example, if a robot can send video and range information, the video and range information can be projected into a 3D space around the robot, rather than being displayed in separate UI windows 
Such a projection allows the user to easily relate visual and range information, and relate that information to the ongoing robot control task, which in turn increases task performance\cite{ricks2004ecological}. . 


\subsubsection{Metrics for the Behavior of Swarms}

In order to determine the quality of the behavior of the swarm, its efficacy at performing tasks must be measured. 
Harriot et al propose that metrics for measuring the interaction of humans and swarms differs significantly from the interaction of humans and individual robots \cite{harriott2014biologically}.
Nine classes of metrics are proposed. 
The most obvious differences from individual robot control metrics are the ideas of leadership within the swarm, and macromovement of the swarm. 
Human attributes - how the human interacts, trust, intervention frequency, etc. \\
Task performance - ability to accomplish task, speed, accuracy, cost \\
Timing - Command diffusion lag, behavior convergence \\
Status - Condition of the swarm, battery life, number of functioning members, stragglers \\
Leadership - Interaction between special members of the swarm and others, how well leaders are followed \\
Decisions - How actions are taken, likelyhood that the correct action is chosen \\
Communication - Speed, range, network efficiency \\
Micromovements - Motion of individual swarm elements relative to each other \\
Macromovement - Motion of the overall swarm, flocking, elongation, forming fancy shapes \\ 
Harriet et al also put the estimated transition point between multi-agent control and swarm around 50 individuals. 
Above that threshold, human interaction may be able to remain focused on macro level behavior, influencing the overall behavior of the swarm rather than control of individuals. 

Emergent behaviors arise from the interactions of actors with each other and the world around them. 
In the face of uncertainty in the world, the behaviors will also become uncertain. \change{cite trust work}
Programs synthesized to guide the swarm should be designed to be robust against failure or degradation of swarm members. 
The heterogeneity of the swarm may also be leveraged to increase its robustness against failures of individual nodes or alterations of the environment. 
Because heterogeneity increases the dimensionality of the solution space for program synthesis, it may adversely affect the performance of the program synthesis and the swarm's runtime convergence to the desired state.

A concept of the swarm as a whole as a programmable entity runs into trouble with reliability. 
In conventional compilation, assuming the compiler is correct and the computer is correct, the compiled binary does what the source code says. 
Robots interact with the real world, which is much less likely to be ``correct'' in the same sense a compiler can be asserted to be. 
Programs for swarms are only going to be functional within some probabilistic grounds and assumed conditions. 
This requirement indicates that the situation has to be at least somewhat known ahead of time, so that the robots will all receive programs that allow them to perform the task.
In the ideal case, the emergent action of all of the robots interacting with the environment causes them to perform the task. 

Swarms have more uncertainty, because the reliability of individual robots is low; and higher attentional demands because there are many robots. 
It may be that above some threshold, the attentional demand will drop again, as the group is no longer treated as a large number of individuals, but as a single group. 
The user interface may be able to drive this re-imagining, and convey other information, by depicting the group in different ways \cite{manning2015heuristic}.
The base case is to simply display all the units as individuals. 
Other approaches include an amorphous shape covering the area occupied by the swarm, an amorphous shape with density shading and motion arrows, the fields of influence for leaders in the swarm, and the web generated by the flow of information within the swarm. 
Considered as a whole, the swarm has properties, such as center of gravity or flock thickness, that do not exist in individual robots. 
Views of these properties may assist the user, for example in determining what areas have insufficient robot density for a thorough search operation. 
%A lot of Dr. Adams' work in HSI was under ONR Award N00014-12-1-0987

If the user is unconcerned with the functioning of individual swarm members, so long as the swarm as a whole remains functional, the UI may simply drop malfunctioning individuals from view. 
This handling of error conditions on individual swarm units fits with the assumption that the swarm as a whole achieves robustness through redundant expendable units, while also allowing the human user to have a rough idea of how the situation is developing by watching the cloud shrink. 
Do long as progress appears to be being made on the mission, the user might let underperforming units slide. 
The supervisory system might not even announce when units are lost, until it starts to affect performance.  
In the limit, the swarm could be treated as a gas, and for tasks such as diffusion over an area, the performance of the swarm would be compared to the behavior of an ideal gas \cite{jantz1997kinetics}.
The addition of sensors and computation would then allow the robots to outperform a gas at tasks, and so achieve higher scores on a task-oriented metric than a gas could attain. 
 
\subsection{Conversion of User Tasks into Robot Programs} 

The current state of the art in design of software for a decentralized swarm controller appears to be one of two methods. \unsure{define why I'm not doing centralized control}
The first method is to come up with behavior primitives that can be performed as a function of local sensing by each member of the swarm. 
These primitives are then composed to come up with a program that the programmer thinks is likely to result in the desired behavior. 
The program is then tested and modified based on the result of the testing in an iterative fashion to converge on the desired behavior. 
Genetic algorithms and similar approaches have been explored for automating the iterative development of composed programs. 
The second method is to define roles for each robot, and a sequence of actions to be performed by each role. 
\unsure{Citations needed, is this as brittle as it sounds? This is where we get markets for selection of roles.}
Neither of these approaches describes communication back to the user, and while programming languages and frameworks can be considered an HRI of sorts, neither of these approaches describes the HRI of the software as it runs. 

In the abstract case, it is clear that any problem that can be solved by agents with homogeneous software can also be solved by agents with heterogenous software. 
Assume that for N agents, there are N distinct programs, one per agent, that can, given each agents' location and state, be executed so that the overall final state of the system arrives at the desired result. 
This is the heterogeneous software case, but can be converted to the homogeneous software case by producing one program, consisting of each of the N distinct programs from the heterogeneous case, plus a conditional control structure that executes the required program based on the robot's location and state.  
From a biomimetic standpoint, this is similar to asking if worker ants learn. 
If they learn, then the ants are hardware-homogeneous and software-heterogeneous, but if the worker ants cannot learn, then they are both hardware and software homogeneous. Either way, they can clearly perform the tasks required of worker ants. 

\subsection{Swarm Software Development Methods}

Because the conversion of the specification of desired behavior for the swarm into individual programs for the swarm controllers is still an open question, it is necessary to understand the current methods used in the development of programs for swarm robots. 
Much of swarm robotic development follows the usual model of software development. 
Starting from a desired functionality, the developer writes a program that they think will provide that functionality.
The program is then tested, in simulation or on real robots, and its behavior is observed. 
The programmer then modifies their program to account for any observed difference between the desired function and the system's behavior. 
This loop of coding, testing, and coding again is repeated until the system behaves as expected, or the programmer graduates. 

\subsubsection {Language Approaches}

One approach to the conversion of the command language to programs for the robots is to define a transformation from the command language to executable code that can be codified into a compiler. 
As a consequence, input in the command language defines a program which is compiled and loaded onto the robots. 

One possibility is to define a command grammar like the programming environment Tierra. \cite{ray1991approach}
In Tierra, there is no such thing as an invalid program. 
All sequences of the existent symbols are regarded as executable programs, although some are more functional than others. 
While this does prevent the possibility of the user becoming frustrated by being told that their commands are invalid, the ease of creating valid, but undesirable commands creates other problems. 
Good user interface designs permit easy undoing of an undesired command, but with the physical world, this can be difficult. 
It is far easier reset a simulation of a robot than it is to recall a real robot from falling down the stairs. 
Furthermore, a command language with no invalid commands runs into the same problem as free-space gesture interfaces: there is no way for a user to be idly present to the interface.
Every motion is interpreted as something that must be acted upon. 
%Is it possible to build a compiler that converts user commands in task space into programs for robots without having to code in assumptions about the qualities of the robots? Does the compiler have to account for global properties (e.g. GPS denial)? Can the compiler account for robot-local properties (e.g. battery life, motor failure)?

\subsubsection{Amorphous Computing}

Amorphous computing (AC), also called spatial computing, is computation using locally-linked and interacting, asynchronous, unreliable computing elements dispersed on a surface or throughout a volume \cite{abelson2000amorphous}. 
The motivation for AC is that while it may be possibly to produce arbitrary quantities of ``smart dust'', it is not possible to ensure that it all works well and is precisely located, especially in real-world applications.
The goal of AC is to get useful work out of such materials, despite uncertainty as to their reliability and location. 
Smart dusts are also the limit-case, in terms of scale, for swarm robotics, and if AC promises to get useful work out of smart dust, then it also has some applicability to larger swarm robots.

There are several languages intended to program amorphous computers. 
"Proto" is a language for a continuous plane spatial computer \cite{correll2009ad}.
Because the devices are distributed over a plane, the difficulty in communicating between any two devices is a function of the distance between them, much as with RF or other radiative communications.
In Proto, the behavior of regions of space is described by the programmer, and the description is transformed into local actions for the network of devices. 
Because devices have a size in the real world, and space between them, the devices cannot not have a one-to-one mapping with the space, but instead perform an approximation of the desired behavior. 
Swarm robots are mobile, so some swarm algorithims can be implemented as a description of constraints on the robot's state, such as "the robot must have communications links to no more than 2 and no less than 1 other robots", and a command to move randomly unless the constraint is satisfied. 
Within a bounded environment, such an algorithm can be shown to converge \cite{correll2009ad}. 

Proto also has considerable appeal as a programming language for swarm control development because of the layered mapping from behavior of regions at the global level to programs for discrete points at the level of individual devices \cite{beal2006infrastructure}. 
If user interface interactions can be interpreted as indications of desired behaviors displayed over spatial regions, then conversion of those behaviors into programs in Proto may be automatable. 
As a result, the user can create control programs for each robot without having to learn Proto themselves.  

Growing Point Language (GPL) allows the specification of topological patterns in an amorphous computer, and so can also be used to specify the distribution of swarm robots in a space \cite{nagpal2004engineering}. 
GPL is inspired by the morphogenic controls present in biological organisms, which use gradients of chemicals called morphogens to dictate the development of cells \cite{turing1952chemical}.
The name GPL arises from one of the language's main abstractions, the growing point. 
Growing points move through the medium, affecting the state of the computational points they pass, and pheremones in the medium which control the motion of growing points.
The abstractions used in GPL are very similar to those that guide the development of biological systems, such as the phototropism of plants and the chemical gradient signaling that coordinates morphogenesis in developing organisms.   
The growing point is the location of activity within the amorphous medium, at which local agents are changing their state. 
Importantly, GPL does not make any prior assumptions on the location of the particles in the system, or robots in the swarm, aside from that they are sufficiently dense in the medium. 
For swarm robotics, this is an important quality, as precise localization may not be available. 
Initially, all agents have the same state and program, with a few exceptions that serve as seeds for the growth to begin. 
If the pattern is not required to be fixed at a particular location, even the seeds could be undetermined initially, and elect themselves via a method such as lateral inhibition. 
During the execution of the GPL program, each agent chooses its state based on the presence of pheromones, which are morphogens with limited range. 
Range limitation on morphogens propagating between robots is set using a TTL (Time To Live) counter that propagates with the morphogen, and is decremented with each hop in the communication network. 
When the TTL hits zero, the morphogen message is no longer propagated. 
By controlling the production or propagation of morphogens within the amorphous medium, complex patterns can be developed. 

Origami Shape Language (OSL) uses the abstraction of a foldable sheet to form shapes, inspired by both origami and the folding of epithelial cells during the development of biological organisms \cite{nagpal2004engineering, nagpal2001programmable}.
Regions and edges on the sheet can be defined by propagation of morphogens, and folds along the edges result in the development of the final form.
Because of the use of morphogens and local communication between the agents on the sheet, there is no need for a global controller to dictate the development of the final form. 
Further, because the high-level description of the desired form does not involve abstractions of the underlying modules, OSL could operate on interlocking modular robots, actuated flexible materials, swarms, or other kinds of computational media. 
In fact, the flexible sheet could be assumed to be virtual, and the resulting motions of the sheet could be translated into motor commands to configure swarm robots into specific arrangements in space. 

\subsubsection{Compositional Approaches}

One possible approach is the composition of programs from behavioral primitives, such that some combination of the primitives results in the emergence of the desired behavior. 
In ``Occlusion-Based Cooperative Transport with a Swarm of Miniature Mobile Robots", individual robots cooperate to push an object to a beacon based on simple behaviors \cite{chen2015occlusion}. 
If the view of the beacon is blocked, the robots push the object. 
If the robots can see the beacon, they wander and avoid obstacles. 
The sum of the two behaviors results in a net pushing force on the side of the object opposite the beacon, which moves the object to the beacon. 

One abstraction of a swarm for programming is a collection of mobile, physically situated processors communicating over an ad-hoc network \cite{evans2000programming}. The behavior of the swarm is emergent from the behavior of individual units, and is resilient against "misbehaving members".
Two approaches to programming this abstraction of a swarm are to compose program from individual behavior primitives, or to synthesize unit programs based on description of environment and desired behavior. 
%Analysis is approximate, rather than general proof of properties.

In the compositional approach, any combination of behaviors is a valid program, but might not be a good/useful program. 
Evans lists disperse (no other nodes within distance d), general disperse (no more than n nodes within distance d), clump/cluster, attract to location, swarm in a direction, and scan area as primitives for motion, and broadcast, partialcast, and unicast as methods of swarm communication.
Behaviors in a program can be composed to operate serially or in parallel, and divided across the swarm or synchronized across the swarm. 
Most biological swarms, such as insects, could be argued to have a priority based parallel evaluation, where different behaviors become active as context requires.
Higher organisms, including humans in large groups, engage in serial and synchronized behaviors. %TODO is there evidence for one kind or the other? 

Another proposed catalog of behaviors for swarm control bases the simple behaviors on pheremones or chemical sensing in single cells \cite{nagpal2004catalog}. 
The proposed behaviors are the use of gradient sensing for position and direction information, local inhibition and competition, lateral inhibition for spatial information, local monitoring, quorum sensing for timing and counting, checkpoint and consensus sensing, and random exploration. 
The first five are common in amorphous computing, but the last three are not. %TODO what are their uses?. 
These senses are sufficient for relatively complex behaviors. 
Quorum sensing is used to detect whether the local agent count is sufficient for a task. 
This allows just-in-time allocation of robots to tasks rather than pre-allocation when the task is designed. 
This may be more robust against failures of individual robots, as it uses the robots that are in the right place at the right time, rather than waiting for specially assigned robots. 
However, under sufficiently bad conditions, a sufficient quorum may never arrive, deadlocking the task. 
In combination with domino timing, where completion of each phase triggers the next, this could then deadlock the entire process unless another mechanism detects and corrects it.

It has been demonstrated that a swarm can perform construction tasks using only local sensing and no communication \cite{wawerla2002collective, bowyer2000automated}.
However, the addition of communication between systems and memory of the state of the world will improve the efficency of the system.
The system under discussion was developed to have the task implied by the behaviors available for the agents, rather than generating the program from a higher-level specification, such as the form of the structure to be built.

Another compositional method for programming robots proposes that the behaviors can be seperated into classes, such as motion, orientation, and so forth \cite{mclurkin2004stupid}. 
Among these behaviors are ``primitives'' such as several forms of clustering, which other, later works have treated as an emergent behavior itself, arising from more primitive primitives. 
The variable granularity of the primitives available to compose swarm control programs seems to point to a hierarchy of control elements, with perhaps single motor operations at the bottom, and an increasing composition of elements to create more and more complex behaviors.
Demos written to run on the swarm called multiple primitive behaviors, with parameters such as degrees of bearing and centimeters of proximity. 
The behaviors ideally run concurrently, and some of them respond to sensor inputs. 
The output of behaviors is whether they are running, translational and rotational velocity, and LED configuration. 
Subsumption and summation are used to arbitrate between behaviors of differing priorities. 

McLurkin's swarm had many properties that would be highly desirable in a swarm intended to operate robustly in a potentially hostile environment. 
For example, the robots did not have unique identities, and so were interchangeable as far as any algorithms involving them were concerned. 
It might be useful for robots to have global, local, or no names, or temporary names, but any algorithm that depends on them may end up depending too much on a particular robot. 
However, the swarm did not have any learning within the robots, and relied on the configuration of the environment to inform the behavior of the robots. 
As a result, ``bad environments" could elicit pathological behavior from the swarm. 
In real use cases, there is no such thing as a ``bad environment". 
The environment simply is, and so if a robot fails in an environment, it is the robot and not the environment that is bad. 

The absence of global IDs for the robots does pose a problem for determining if all of the robots are reachable from all of the other robots, in other words, if the network is a single connected component. 
With global IDs, the absence of some subset of them indicates that the network is divided. 
Without global IDs, the network may be divided but the robots will be unable to detect it. 

Unfortunately, much of the design of programs for swarm robots consists of iterating between coding and observing behavior of the system in an ad hoc process \cite{palmer2005behavioral}. 
Emergence is the transition between two levels of behavior. 
At the base level is the behavior explicit in the program and the actions generated by it in the environment. 
Emergent behaviors are those behaviors of a system that are irreducible to the actions of a single individual, but arise from the interactions of multiple individuals with each other and the environment \cite{tsuda2004robust}.
At the higher, emergent level is the implicit behavior of the program, as the sum of the results of the robots' actions. 
Babarilla et al propose a taxonomy of behaviors that separates spatially organizing, navigation, decision-making, and ``other'' behaviors into their own groups \cite{brambilla2013swarm}. 
These behaviors are higher level, swarm-wide behaviors, rather than individual behaviors of individual robots. 
If it is assumed that these are the emergent behaviors of the system, then perception of the emerging behavior can in turn affect the base level behavior of the robots, providing feedback. 
This feedback can be used to increase the tendency of the behavior of the interacting swarm members to converge on a solution. 
By observing the state of the environment, swarm robots determine if they are satisfied with the local state. 
The dissatisfied members change their actions, but the satisfied units do not. 
However, this feedback alone will not stabilize, as environmental changes are not required to increase the total of satisfied robots. 
The longer a unit remains satisfied, the less easy it is for it to become dissatisfied. 
This bias helps prevent the disruption of partial solutions as the situation develops. 
However, it does require a local criteria for detection of a partial solution, and so cannot be used to solve problems whose progress is not locally visible. 

These emergent approaches do not have the robots perform all of their available actions all of the time. 
Instead, it is assumed that the behavior of each robot is controlled by its reaction to the environment around it, and possibly to signals from other robots, so that actions are only performed when they are required. 
As a result, user programs compiled from a higher-level representation could be a table consisting of possible values for the sensors, and the actions to undertake when those values are met.
``A Compositional Framework for Programming Stochastically Interacting Robots" \cite{napp2011compositional} provides a formal framework for the analysis of this type of compositional program, called Guarded Command Programming with Rates (GCPR). 
Robots are assumed to only have local sensing.
The guards of GCPR are conditions on the environment.
When a condition is met, the robot performs actions at a given rate. 
%This is a lot like behavior-based robotics.
In the concurrent case, this is modeled as each action happening one at a time, but in random order. 
On a real swarm, the actions would take place in parallel, but the concurrent model is more amenable to analysis. 
To determine if a set of actions will be successful, it is required to ensure that for all orderings of all actions, the final state space of the swarm is the desired final state. 
Failed behaviors are modeled as a program that doesn't do what the other program it is composed with does. 
Correct programs are those that reach the target state with probability one, even when composed with bounded failures. 
Once the target state is reached, the program is assumed to alt, so while the final state may be reached very slowly, but once it is reached, it is not left. 
In the GCPR models, the time to execution of an action is stochastic, but in the real-world case of noisy or imperfect sensors, the variable time to execution of a guarded behavior would be caused by the imperfection of the robot's ability to detect that the guard was satisfied. 
In the real world, we get stochasticity for free. 

%$\Rightarrow$ Get Klavins et al 2k4, apparently on algorithmic approach mapping local to complex global behavior.

\subsubsection{Evolutionary Composition}

Determining how the behaviors should be composed for an individual swarm robot's controller is difficult. 
One approach is to set some intuitively reasonable values and then test the behavior of the software on the swarm. 
This may not scale well to large swarms, depending on the technology used to distribute the software\change{cite nagpal and kilobots}. 
If each robot needs to be individually reprogrammed, then the time required for distributing a software update will grow with the size of the swarm, and eventually become unmanageable. 
Assuming the program can be swiftly distributed, then programming it is simply a matter of iterative software development, with each bug or undesired behavior accounted for in the next iteration. 
However, this approach assumes that the behavior of the swarm is being dictated by programmers, and expressed as a computer program. 
While the swarm's behavior must be guided by programs at some level, the overall behavior of the swarm should be possible for non-programming users to express and control. 

Genetic algorithims (GA) have been proposed as a possible way of directing the composition of behavioral primitives into programs \cite{palmer2005emergence}.
A GA expresses the robot program as a genome, which is translated into the actual program and run on the robot. 
The result of each program is assessed using a fitness function and then the genomes for the best programs are combined to produce a new generation of genomes. 
This cycle of combining and assessing genomes continues until a certain level of quality is reached, as judged by the fitness function.

However, it is difficult to automatically extract from the known behavior of the system an overall understanding of the progress it is making on the task, and so a value for output of the fitness function. 
Worse, without a time bound on solving a problem or a way to calculate progress, it is impossible to tell if a program has failed, or has merely not succeeded yet.
For example, assume a program's intended purpose is to gather units of a resource at a goal. 
If the program merely moves the units stochastically, sometimes they will enter the goal, creating an appearance of progress. 
However, it may be vanishingly unlikely that all the units will randomly happen to be in the goal at once. 
Even a program that cannot find the goal, and so will never put any units in it, behaves exactly the same as a perfect resource-gathering program that just hasn't moved any units to the goal yet. 
Ideally, it would be possible to recognize and evaluate performance on sub-problems. 
Palmer proposes that the interactions and emergent behavior of the system are observable, while the reactions of the agents in the system are programmable, and so by observing the interactions and emergent behavior, the developer can receive feedback on how the system is progressing. 
This emergent behavior is discovered initially in simulation, although getting groups of people to perform the task and observing their interactions is also suggested as a possible initial mechanism.
While these levels of emergence and reaction can in theory be nested to any depth, it is possible to treat each pair of observable features and reactions to them as independent, because the only explicit causation is in the reactions. 
The emergent behavior is not explicitly defined, and so does not have a direct causal link to the next level up. 
However, all of the proposed observation and hierarchy is intended as a design process, not an automation process. 
In other words, while it is a structure for guiding the development of an emergent system, the system is still developed by programmers writing code and then running it on the robots.

It is possible to develop self-organizing behaviors for robots using genetic algorithms \cite{dorigo2004evolving}. 
This does offer a reduction in the amount of hands-on time spent programming, but it frequently comes at the expense of time spent waiting for the system to converge, or determining why it converged on a problematic solution. 
Performing the evolution simulation allowed the robots to develop behaviors such as aggregation, but early versions also allowed the evolved motion strategy to acquire a high fitness by spinning in place. 
For mutual motion when linked, robots could sense traction, allows negotiation of a common direction of movement by proprioception. 
Similar traction-based interactions allowed the robots to avoid obstacles and pull objects. 
Solutions discovered by genetic algorithms are also prone to overfitting. 
The swarms described in Dorigo et al decreased in performance when the number of robots involved in the swarm was changed from the values used to evolve the solutions, and when a more accurate physical model was used in the simulations.

Aggregation behavior has also been evolved in swarm robots, using a perceptron as the controller for the robots and evolving the perceptron weight vector \cite{bahgecci2005evolving}. 
Aggregation was chosen because it is a preliminary behavior primitive, which the swarm might engage in prior to doing some other task, such as moving an object, attacking \emph{en mass}, etc.
The resulting vector only controls aggregation behavior, so each behavioral primitive would require its own evolutionary development. 
Evolution is only useful if it is a faster approach than designing the behavior by hand. 

Quinn et al have developed controllers that allow robots to move into formation from random starting positions \cite{quinn2003evolving}. 
These controllers use local interactions and minimal sensing to achieve their goals. 
One point the authors make, which is not frequently mentioned in other work, is that while flocking or shoaling behavior is a relatively simple behavior to have emerge from robots who can detect the distance, postion, and velocity of the other nearby robots, implementing that perception on real robots is quite difficult.
Because a specific behavior was desired, the fitness function used to evolve it was specified in terms of metrics related to the behavior. 
Task-specific fitness functions are also found in later work on evolution of swarm robot behavior, which seems to indicate that evolution of behaviors in swarm robots may only be a time-complexity tradeoff. 
The complexity of directly specifying the task is reduced to the complexity of describing the results in the fitness function, but the time required increases to that required by the evolutionary iterations. 
Additionally, the ad hoc iterative process of creating emergent behaviors is replaced by an ad hoc iterative process of creating fitness functions.
Developing novel behavior in the field by converting user specifications of the behavior of the swarm into a fitness function for a genetic algorithm are unlikely to yield results in a timely manner. 

Interestingly, some of the work in evolvable controllers leads to inter-robot communication as one of the emergent properties of the evolved controller \cite{quinn2001evolving}.
In order to move as a formation, one of the robots must be the leader, but there is nothing in the fitness function or any of the other code that designates roles for the robots. 
Instead, the selection of the leader arises from the evolutionary development of the controllers, and is present in the controller as a response to a particular series of stimuli. 
Genomes that did not encode such a symmetry-breaking reaction never developed a leader-follower distinction, and so failed to move in formation, and so received low fitness scores. 
For the follow-the-leader task, genetic variation among the robots increased fitness more readily than having all robots share the same genome \cite{quinn2001comparison}.
The condition where all robots shared the same genes was called ``clonal", while each robot having its own genome was ``aclonal".
Oddly, while one would expect that the aclonal condition would result in a specialization, with each robot developing a genome that performed either the leader or follower role well, the aclonal condition developed robots which could perform both roles. 
Quinn hypothesizes that while the clonal condition had to evolve roles and an allocation mechanism simultaneously, the aclonal condition could specialize the roles during early evolution, and then develop an arbitration mechanism to select roles.

\subsubsection{Pheromone Approaches}

One possible approach to a control UI for a remotely-located swarm is a multi-touch interface for specifying a vector field \cite{Kato:2009:MIC:1520340.1520500}.
Because the user interface design focuses on the vector field rather than individual robots, the same control interface can scale to an arbitrarily large collection of robots. 
Vector field paths can have loops, which do not exist in waypoint-based paths. 
Waypoint paths have explicit ends, unless an additional command is added to join beginning and ending points. 
Vector field paths do have some limitations, however. 
Because the vectors are bound to a 2-D plane, the paths they create cannot cross each other. 
Instead, they flow together. 
The vector field can be viewed as an abstraction of pheremone control, or even implemented in terms of the presence or absence of virtual pheremones, but it has some limitations that pure pheremone control does not have.

This user interface does not directly map to programs on the robots. 
Instead, the central computer maintains the vector field representation and commands the individual robots.
This interface does not allow the assignment of tasks to robots, but allows the user to directly control the motion of the robots. 
In order to convert from a task-based user interface to a vector field representation, the field would need to be mapped to the task space. 
Since the robots may not have accurate localization within the task space, it may not be possible to guide the robots by relating their position to a global vector field. 

The use of co-fields may provide a way to move the vector field representation from the central computer to the swarm, or allow the swarm to act for some time without constant updates from a central controller \cite{mamei2003co}.
Co-fields distribute the data within a space, which may be physical or may be abstract. 
Agents react to gradients in field, and spread their own fields over local communication networks. 
The overall vector space created by the user (the UI vector space) could be propagated to the robots periodically, and combined with their own internal vector fields to generate movement based on both the user's desires and the local rules operating on each robot. 
As with general vector fields, knowing which areas of the UI vector space are relevant to each robot may require global localization, and so only be available for swarms operating in conditions that permit global localization. 

Pheromone robotics refers to a metaphor for developing control software for swarm robots. 
Some social animals, especially insects, use chemical signals called pheromones to communicate with each other. 
For example, wasps inside their nest react to the scent of wasp venom by traveling to the outer surface of the nest and attacking nearby moving targets \cite{jeanne1981alarm}.
Ants leave trails of pheremones for other ants to follow to food sources. 
Each individual ant's contribution to the trail can be modulated by the quality of the food source, which allows the reaction of the other ants to the trail to cause an emergent distribution of the foraging work force that favors higher-quality food sources \cite{sumpter2003nonlinearity}.

Both global vector fields and the global and local blending of vector fields in co-fields can be viewed as subsets of pheromone robotics that use a global spatial representation. 
However, it is possible to develop emergent control systems without a global spatial representation.

In pheromone robotics, the pheromones are usually simulated or ``virtual'' pheromones, rather than real chemicals which are detected by chemical sensors. 
Each pheremone can have properties such as diffusion and evaporation rates that result in the pheremone spreading in space or gradually disappearing. 
In addition to its properties, the pheromones may have other characteristics which robots can sense. 
For example, a robot may emit a pheromone which diffuses into the environment and evaporates quickly, so distance from the robot can be determined by the strength of the pheromone, and approaching or avoiding the robot may be accomplished by moving up or down the gradient of pheromone strength. 
If the swarm is engaged in a search, each searching robot may emit a ``search marker'' pheromone that lingers in the area after the robot leaves. 
Other robots, on entering the area, would detect the pheromone and know that searching this area again would be fruitless. 
If the object of the search can move, the marker pheromone could diminish as a function of time, so areas that have not been searched for a long time become unmarked and may be searched again. 
Once the target is found, the robot may stop and emit a ``discovery pheromone'' diffuses into the environment, attracts other robots, and causes them to also emit a discovery pheromone. 
As a result, once any robot discovers the target, all of the robots quickly converge on its location. 

The addition of directional communication for the messages that convey virtual pheromone information allows easy determination of the direction of pheremone gradients \cite{payton2001pheromone}.
Rather than directly diffusing in the space as a chemical would, hop counts in the network of robots simulate diffusion. 
Because routes may be of different length, the message with the lowest hop count is assumed to be the truest indication of minimal distance within the network. 
Payton provides algorithms for budding with growth inhibition to allow a swarm to explore while remaining connected as a network. 
Rather than modeling the world based on the incoming messages, the content of the pheromone messages and the network behavior as a whole serves as a model of the world, mapped 1:1 onto the real environment. 
While it is possible to build a set of behavioral primitives out of pheromone signaling and associated behaviors, controlling the swarm to perform a task with these primitives is still done by hand \cite{payton2003compound}.

In all of these examples, the sensing of the pheremones is assumed to be local to the robot, at least metaphorically. 
To actually maintain pheromones in the environment without robots being present to transmit them requires, again, a global representation of the task space which the robots can refer to when needed. 
In a real implementation, some robots could remain stationary and only act as transponders, computing and transmitting the local pheromone information for a given area. 
However, even if there is no global map and no transponders, and so the robots are limited to only the pheromones they can directly perceive and emit at the present instant, some emergent behaviors are still possible. 

Pheromone approaches can guide the construction of objects, even if the individual swarm members have no memory and only local perception\cite{mason2003programming}. 
The agents engaged in the construction move at random, and take actions governed by their individual perception of environment at present time. 
The agents can release and react to pheromones in the environment, and so there is an implicit communication via stigmurgy, but no explicit agent-to-agent communication. 
A set of environmental triggers is "coherent if no stage in the building process can be confused with an earlier stage by making only local observations, thus obviating the need for centralized control".   
By specifying such a coherent set of triggers, the specification for the building can be expressed to the swarm. 
However, Mason does not propose a user interface for creating this specification, saying instead that ``other future work includes programming construction swarms by specifying the target structure directly, letting a compiler infer the corresponding rule-set (if one exists)."

Because pheremones are virtual chemicals with spatial information, it would be possible to combine the use of pheremones with reaction diffusion equations to structure activity within a space or to converge to patterns of activity over time \cite{turing1952chemical}. 
Assuming even diffusion of the robots in space, the global map of the pheremone concentrations is represented over the network by the locally-computed concentrations computed by each robot.   \change{also develop this para}

\section{Conclusion}

The various approaches to development of swarm robot control programs show that a wide variety of approaches can still result in robust controllers for swarm robots. 
Previous work in HRI shows that multi-touch interfaces will allow a scalable and direct mapping between the desires of the user and sequences of commands to the swarm. 
While swarm hardware is not yet at a point where very complex computation may be pushed directly to the swarm nodes themselves, that time is not far off. 
Until computational power in the individual swarm units does reach the levels required for complex computation, virtualization of computing resources can provide an adequate test environment for the development of swarm control algorithms at modest requrements in terms of space and power consumption. 
\unsure{develop conclusion more, perhaps restate some of research questions?}

\bibliography{swarm.bib}
\bibliographystyle{plain}

\end{document}

%Papers to read

%Cooperative interaction of walking human and distributed robot maintaining stability of swarm 

%Development of IR-based short-range communication techniques for swarm robot applications 

%The Wanda Robot and Its Development System for Swarm Algorithms

%Stability of swarm robot based on local forces of local swarms

%Swarm robot pattern formation using a morphogenetic multi-cellular based self-organizing algorithm 

%A particle-swarm-optimized fuzzy-neural network for voice-controlled robot systems 

%The I-SWARM project 


